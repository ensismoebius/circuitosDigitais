\documentclass[aspectratio=169,dvipsnames,table]{beamer}

% Basic packages 

% Absolute position for text and elements
\usepackage[absolute,overlay]{textpos}

% Enables text replacing
\usepackage{xstring}

% Enable long tables
\usepackage{longtable}

% Enable cvs data load
\usepackage{csvsimple, booktabs}

% Allow more floats whatever it is
\usepackage{morefloats}

% Use latim special characters
\usepackage[utf8]{inputenc}

% Use colors
\usepackage{color}

% Enables landscape pages
\usepackage{lscape}

% Enables graphics
\usepackage{graphicx}

% Better text jstification
\usepackage{microtype}

% Indent the section first paragraph.
\usepackage{indentfirst}

% Enables code listing
\usepackage{listings}

% Enables multicolumn text
\usepackage{multicol}

% Enables math
\usepackage{amsthm}
\usepackage{amsmath}
\usepackage{amssymb}

% Adds some symbols (like slash on a non mathematical context)
\usepackage{textcomp}

% Enables the use of subfigures
\usepackage{subfig}

% Float barriers, prevents figures floats accross
% an determined \FloatBarrier
\usepackage[section]{placeins}

% Enables the use of landscape pages
\usepackage{lscape}
\usepackage{pdflscape}

% Quote packages
% ABNT standard
\usepackage[alf]{abntex2cite}

% Portuguese-specific commands
\usepackage[portuguese]{babel}

% Bibliography 
\usepackage[brazilian,hyperpageref]{backref}

% Drawing and plot packages
\usepackage{relsize}
\usepackage{tikz}

% Enable some musical notation
\usepackage{musicography}

% Enables tiks to draw some circuits plus use some positioning commands
\usetikzlibrary{circuits.logic.US, positioning}

\usetikzlibrary{calc}
\usetikzlibrary{datavisualization}
\usetikzlibrary{positioning}
\usetikzlibrary{mindmap}
\usetikzlibrary{snakes}
\usetikzlibrary{shapes}
\usetikzlibrary{decorations.pathreplacing}
\usetikzlibrary{spy}
\usetikzlibrary{backgrounds}
\usetikzlibrary{patterns}

\usepackage{pgfplots}
\usepackage{pgfplotstable}
\usepgfplotslibrary{units}
\pgfplotsset{compat=newest}

% Packages configuration

% Long tables pre and post spacing
\setlength{\LTpre}{0pt}
\setlength{\LTpost}{-30pt}

\setbeamercolor{background canvas}{bg=}
\setbeamercolor{normal text}{bg=black!20}
\setbeamercolor{title}{parent=structure,bg=cyan}
\setbeamercolor{frametitle}{parent=structure,bg=cyan}

% Enables section slides
\AtBeginSection[]{
	\begin{frame}
		\vfill
		\centering
		\begin{beamercolorbox}[sep=18pt,center,shadow=true,rounded=true]{title}
			\usebeamerfont{title}\insertsectionhead\par
		\end{beamercolorbox}
		\vfill
	\end{frame}
}

% Code listing settings
% Translate Listing -> Algoritmo
\renewcommand{\lstlistingname}{Algoritmo}
% Translate List of Listings -> Lista de Algoritmos
\renewcommand{\lstlistlistingname}{Lista de \lstlistingname}

% The code style
\definecolor{codegreen}{rgb}{0,0.6,0}
\definecolor{codegray}{rgb}{0.5,0.5,0.5}
\definecolor{codepurple}{rgb}{0.58,0,0.82}
\definecolor{backcolour}{rgb}{0.95,0.95,0.92}

% Defines code listing style
\lstdefinestyle{mystyle}{
	backgroundcolor=\color{backcolour},   
	commentstyle=\color{codegreen},
	keywordstyle=\color{magenta},
	numberstyle=\tiny\color{codegray},
	stringstyle=\color{codepurple},
	basicstyle=\tiny,
	breakatwhitespace=false,         
	breaklines=true,                 
	captionpos=b,                    
	keepspaces=true,                 
	numbers=left,                    
	numbersep=2pt,                  
	showspaces=false,                
	showstringspaces=false,
	showtabs=false,                  
	tabsize=2
}
\lstset{style=mystyle}

% Solves annoying problem: Unicode character ​ (U+200B) not set up for use with LaTeX. 
\DeclareUnicodeCharacter{200B}{{\hskip 0pt}}