\documentclass{beamer}

\usepackage{amsmath}
\usepackage{tikz}
\usetikzlibrary{circuits.logic.US}

\title{Exercícios de Circuitos Combinacionais}
\author{Seu Nome}
\date{\today}

\begin{document}
	
	\frame{\titlepage}
	
	% Exercício 1
	\begin{frame}
		\frametitle{Exercício 1}
		Identifique a tabela verdade para a porta lógica AND.
	\end{frame}
	
	\begin{frame}
		\frametitle{Resposta do Exercício 1}
		A tabela verdade para a porta AND é:
		\begin{center}
			\begin{tabular}{|c|c|c|}
				\hline
				A & B & F \\
				\hline
				0 & 0 & 0 \\
				0 & 1 & 0 \\
				1 & 0 & 0 \\
				1 & 1 & 1 \\
				\hline
			\end{tabular}
		\end{center}
	\end{frame}
	
	% Exercício 2
	\begin{frame}
		\frametitle{Exercício 2}
		Dada a tabela verdade a seguir, construa a expressão lógica correspondente:
		\begin{center}
			\begin{tabular}{|c|c|c|}
				\hline
				A & B & F \\
				\hline
				0 & 0 & 0 \\
				0 & 1 & 1 \\
				1 & 0 & 1 \\
				1 & 1 & 0 \\
				\hline
			\end{tabular}
		\end{center}
	\end{frame}
	
	\begin{frame}
		\frametitle{Resposta do Exercício 2}
		A expressão lógica correspondente é:
		\[
		F = \overline{A \oplus B}
		\]
	\end{frame}
	
	% Exercício 3
	\begin{frame}
		\frametitle{Exercício 3}
		Dada a expressão lógica \( F = A \cdot \overline{B} \), desenhe o circuito combinacional correspondente.
	\end{frame}
	
	\begin{frame}
		\frametitle{Resposta do Exercício 3}
		O circuito combinacional correspondente é:
		\begin{center}
			\begin{tikzpicture}
				\node (A) at (0,1.5) {A};
				\node (B) at (0,0) {B};
				\node[not gate US, draw, logic gate inputs=nn] at (2,0) (notB) {};
				\node[and gate US, draw, logic gate inputs=nn] at (4,0.75) (and1) {};
				
				\draw (A) -| (and1.input 1);
				\draw (B) -- (notB.input);
				\draw (notB.output) -| (and1.input 2);
				\draw (and1.output) -- ++(1,0) node[right] {F};
			\end{tikzpicture}
		\end{center}
	\end{frame}
	
	% Exercício 4
	\begin{frame}
		\frametitle{Exercício 4}
		Dado o circuito combinacional a seguir, construa a tabela verdade correspondente:
		\begin{center}
			\begin{tikzpicture}
				\node (A) at (0,2) {A};
				\node (B) at (0,0) {B};
				\node[and gate US, draw, logic gate inputs=nn] at (2,1) (and1) {};
				\node[not gate US, draw, logic gate inputs=nn] at (4,1) (not1) {};
				
				\draw (A) -| (and1.input 1);
				\draw (B) -| (and1.input 2);
				\draw (and1.output) -- (not1.input);
				\draw (not1.output) -- ++(1,0) node[right] {F};
			\end{tikzpicture}
		\end{center}
	\end{frame}
	
	\begin{frame}
		\frametitle{Resposta do Exercício 4}
		A tabela verdade correspondente é:
		\begin{center}
			\begin{tabular}{|c|c|c|}
				\hline
				A & B & F \\
				\hline
				0 & 0 & 1 \\
				0 & 1 & 1 \\
				1 & 0 & 1 \\
				1 & 1 & 0 \\
				\hline
			\end{tabular}
		\end{center}
	\end{frame}
	
	% Exercício 5
	\begin{frame}
		\frametitle{Exercício 5}
		Dada a expressão lógica \( F = A \cdot B + \overline{A} \cdot \overline{B} \), construa a tabela verdade correspondente.
	\end{frame}
	
	\begin{frame}
		\frametitle{Resposta do Exercício 5}
		A tabela verdade correspondente é:
		\begin{center}
			\begin{tabular}{|c|c|c|}
				\hline
				A & B & F \\
				\hline
				0 & 0 & 1 \\
				0 & 1 & 0 \\
				1 & 0 & 0 \\
				1 & 1 & 1 \\
				\hline
			\end{tabular}
		\end{center}
	\end{frame}
	
	% Exercício 6
	\begin{frame}
		\frametitle{Exercício 6}
		Identifique as portas lógicas utilizadas no seguinte circuito combinacional:
		\begin{center}
			\begin{tikzpicture}
				\node (A) at (0,2) {A};
				\node (B) at (0,0) {B};
				\node[or gate US, draw, logic gate inputs=nn] at (2,1) (or1) {};
				\node[not gate US, draw, logic gate inputs=nn] at (4,1) (not1) {};
				
				\draw (A) -| (or1.input 1);
				\draw (B) -| (or1.input 2);
				\draw (or1.output) -- (not1.input);
				\draw (not1.output) -- ++(1,0) node[right] {F};
			\end{tikzpicture}
		\end{center}
	\end{frame}
	
	\begin{frame}
		\frametitle{Resposta do Exercício 6}
		As portas lógicas utilizadas são:
		\begin{itemize}
			\item Porta OR
			\item Porta NOT
		\end{itemize}
	\end{frame}
	
	% Exercício 7
	\begin{frame}
		\frametitle{Exercício 7}
		Dada a expressão lógica \( F = A + B \cdot \overline{C} \), desenhe o circuito combinacional correspondente.
	\end{frame}
	
	\begin{frame}
		\frametitle{Resposta do Exercício 7}
		O circuito combinacional correspondente é:
		\begin{center}
			\begin{tikzpicture}
				\node (A) at (0,3) {A};
				\node (B) at (0,1.5) {B};
				\node (C) at (0,0) {C};
				\node[not gate US, draw, logic gate inputs=nn] at (2,0) (notC) {};
				\node[and gate US, draw, logic gate inputs=nn] at (4,1.5) (and1) {};
				\node[or gate US, draw, logic gate inputs=nn] at (6,1.5) (or1) {};
				
				\draw (A) -- (or1.input 1);
				\draw (B) -| (and1.input 1);
				\draw (notC.output) -| (and1.input 2);
				\draw (C) -- (notC.input);
				\draw (and1.output) -- (or1.input 2);
				\draw (or1.output) -- ++(1,0) node[right] {F};
			\end{tikzpicture}
		\end{center}
	\end{frame}
	
	% Exercício 8
	\begin{frame}
		\frametitle{Exercício 8}
		Construa a expressão lógica a partir do seguinte circuito combinacional:
		\begin{center}
			\begin{tikzpicture}
				\node (A) at (0,2) {A};
				\node (B) at (0,0) {B};
				\node[and gate US, draw, logic gate inputs=nn] at (2,1) (and1) {};
				\node[or gate US, draw, logic gate inputs=nn] at (4,1) (or1) {};
				
				\draw (A) -| (and1.input 1);
				\draw (B) -| (and1.input 2);
				\draw (A) -- ++(0.5,0) |- (or1.input 1);
				\draw (and1.output) -- (or1.input 2);
				\draw (or1.output) -- ++(1,0) node[right] {F};
			\end{tikzpicture}
		\end{center}
	\end{frame}
	
	\begin{frame}
		\frametitle{Resposta do Exercício 8}
		A expressão lógica correspondente é:
		\[
		F = A + A \cdot B
		\]
	\end{frame}
	
	% Exercício 9
	\begin{frame}
		\frametitle{Exercício 9}
		Dada a expressão lógica \( F = (A \cdot B + \overline{C}) \cdot (A + C) \), desenhe o circuito combinacional correspondente.
	\end{frame}
	
	\begin{frame}
		\frametitle{Resposta do Exercício 9}
		O circuito combinacional correspondente é:
		\begin{center}
			\begin{tikzpicture}
				\node (A) at (0,4) {A};
				\node (B) at (0,2) {B};
				\node (C) at (0,0) {C};
				\node[not gate US, draw, logic gate inputs=nn] at (2,0) (notC) {};
				\node[and gate US, draw, logic gate inputs=nn] at (4,3) (and1) {};
				\node[or gate US, draw, logic gate inputs=nn] at (6,3) (or1) {};
				\node[or gate US, draw, logic gate inputs=nn] at (4,1.5) (or2) {};
				\node[and gate US, draw, logic gate inputs=nn] at (8,3) (and2) {};
				
				\draw (A) -| (and1.input 1);
				\draw (B) -| (and1.input 2);
				\draw (and1.output) -- (or1.input 1);
				\draw (C) -- (notC.input);
				\draw (notC.output) -| (or1.input 2);
				\draw (A) -- ++(2,0) |- (or2.input 1);
				\draw (C) -- ++(2,0) |- (or2.input 2);
				\draw (or1.output) -- (and2.input 1);
				\draw (or2.output) -- (and2.input 2);
				\draw (and2.output) -- ++(1,0) node[right] {F};
			\end{tikzpicture}
		\end{center}
	\end{frame}
	
	% Exercício 10
	\begin{frame}
		\frametitle{Exercício 10}
		Dado o circuito combinacional a seguir, construa a expressão lógica e a tabela verdade correspondente:
		\begin{center}
			\begin{tikzpicture}
				\node (A) at (0,4) {A};
				\node (B) at (0,2) {B};
				\node (C) at (0,0) {C};
				\node[or gate US, draw, logic gate inputs=nn] at (2,3) (or1) {};
				\node[and gate US, draw, logic gate inputs=nn] at (4,1) (and1) {};
				\node[and gate US, draw, logic gate inputs=nn] at (6,2) (and2) {};
				\node[or gate US, draw, logic gate inputs=nn] at (8,2) (or2) {};
				
				\draw (A) -- (or1.input 1);
				\draw (B) -| (or1.input 2);
				\draw (or1.output) -| (and1.input 1);
				\draw (C) -- (and1.input 2);
				\draw (A) -| (and2.input 1);
				\draw (and1.output) -- ++(0.5,0) |- (and2.input 2);
				\draw (and2.output) -- (or2.input 1);
				\draw (B) -| (or2.input 2);
				\draw (or2.output) -- ++(1,0) node[right] {F};
			\end{tikzpicture}
		\end{center}
	\end{frame}
	
	\begin{frame}
		\frametitle{Resposta do Exercício 10}
		A expressão lógica correspondente é:
		\[
		F = (A + B) \cdot C + A \cdot ((A + B) \cdot C) + B
		\]
		A tabela verdade correspondente é:
		\begin{center}
			\begin{tabular}{|c|c|c|c|}
				\hline
				A & B & C & F \\
				\hline
				0 & 0 & 0 & 0 \\
				0 & 0 & 1 & 0 \\
				0 & 1 & 0 & 1 \\
				0 & 1 & 1 & 1 \\
				1 & 0 & 0 & 0 \\
				1 & 0 & 1 & 1 \\
				1 & 1 & 0 & 1 \\
				1 & 1 & 1 & 1 \\
				\hline
			\end{tabular}
		\end{center}
	\end{frame}
	
\end{document}
