\documentclass [15pt,a4paper,twoside]{article}
\usepackage[portuges,shorthands=off]{babel}        % shorhands=off is required for babel french in combination with tikz karnaugh....
\usepackage[utf8x]{inputenc}
\usepackage[T1]{fontenc}
\usepackage{amsmath}
\usepackage{geometry}
\geometry{verbose,a4paper, tmargin=3.5cm,bmargin=3.5cm,lmargin=2.5cm,rmargin=2.5cm,headsep=1cm,footskip=1.5cm}
\usepackage{fancyhdr}
\usepackage{colortbl}
\usepackage[dvipsnames]{xcolor}
\usepackage{tikz -timing}
\usepackage{tikz}
\usetikzlibrary{karnaugh}
\pagestyle{fancy}

\definecolor{LogisimKMapColor0}{RGB}{128,0,0}
\definecolor{LogisimKMapColor1}{RGB}{230,25,75}
\definecolor{LogisimKMapColor2}{RGB}{250,190,190}
\definecolor{LogisimKMapColor3}{RGB}{170,110,40}
\definecolor{LogisimKMapColor4}{RGB}{245,130,48}
\definecolor{LogisimKMapColor5}{RGB}{255,215,180}
\definecolor{LogisimKMapColor6}{RGB}{128,128,0}
\definecolor{LogisimKMapColor7}{RGB}{255,255,25}
\definecolor{LogisimKMapColor8}{RGB}{210,245,60}
\definecolor{LogisimKMapColor9}{RGB}{0,0,128}
\definecolor{LogisimKMapColor10}{RGB}{145,30,180}
\definecolor{LogisimKMapColor11}{RGB}{60,180,175}
\definecolor{LogisimKMapColor12}{RGB}{0,130,203}
\definecolor{LogisimKMapColor13}{RGB}{230,190,255}
\definecolor{LogisimKMapColor14}{RGB}{170,255,195}
\definecolor{LogisimKMapColor15}{RGB}{240,50,230}

\fancyhead{}
\fancyhead[C] {Logisim-evolução gerada pelo documento em Mon Aug 12 01:57:27 BRT 2024}
\fancyfoot[C] {\thepage}
\renewcommand{\headrulewidth}{0.4pt}
\renewcommand{\footrulewidth}{0.4pt}

\makeatother

\begin{document}
\section{Introdução}
Este documento foi gerado por logisim-evolution. Qualquer parte das fontes do TeX pode ser usada em seus próprios documentos sem nenhum problema. Caso você queira usar todas/partes das fontes TeX geradas, por favor (1) não se esqueça de incluir os pacotes necessários, e (2) inclua uma observação de que esta fonte foi gerada pela logisim-evolução.
%===============================================================================
\section{Tabela da verdade}
A tabela pode ser muito grande para ser exibida na página. No tempo de geração não foi feito nenhum cálculo sobre o tamanho da tabela em relação à largura/altura da página.
%-------------------------------------------------------------------------------
\subsection{Tabela da verdade compactada}
\begin{center}
\begin{tabular}{cc|c}
$x$&$y$&$z$\\
\hline
$-$&$0$&$1$\\
$0$&$1$&$0$\\
$1$&$1$&$1$\\

\end{tabular}
\end{center}
%-------------------------------------------------------------------------------
\subsection{Tabela da verdade completa}
\begin{center}
\begin{tabular}{cc|c}
$x$&$y$&$z$\\
\hline
$0$&$0$&$1$\\
$0$&$1$&$0$\\
$1$&$0$&$1$\\
$1$&$1$&$1$\\

\end{tabular}
\end{center}
%===============================================================================
\section{Diagramas de Karnaugh}
Esta secção mostra várias versões dos diagramas de Karnaugh das funções dadas.
%-------------------------------------------------------------------------------
\subsection{Diagramas de Karnaugh vazios}
\begin{center}
\begin{tikzpicture}[karnaugh,disable bars,x=1\kmunitlength,y=1\kmunitlength,kmbar left sep=1\kmunitlength,grp/.style n args={4}{#1,fill=#1!30,minimum width= #2\kmunitlength,minimum height=#3\kmunitlength,rounded corners=0.2\kmunitlength,fill opacity=0.6,rectangle,draw}]
\karnaughmap{2}{$z$}{{$x$}{$y$}}{}{
\draw[kmbox] (-0.5,2.5)
   node[below left]{$x$}
   node[above right]{$y$} +(-0.2,0.2)
   node[above left]{$z$};\draw (0,2) -- (-0.7,2.7);
\foreach \x/\1 in %
{0/0,1/1} {
   \node at (\x+0.5,2.2) {\1};
}
\foreach \y/\1 in %
{0/0,1/1} {
   \node at (-0.4,-0.5-\y+2) {\1};
}
}
\end{tikzpicture}
\end{center}
%-------------------------------------------------------------------------------
\subsection{Preenchido com diagramas de Karnaugh}
\begin{center}
\begin{tikzpicture}[karnaugh,disable bars,x=1\kmunitlength,y=1\kmunitlength,kmbar left sep=1\kmunitlength,grp/.style n args={4}{#1,fill=#1!30,minimum width= #2\kmunitlength,minimum height=#3\kmunitlength,rounded corners=0.2\kmunitlength,fill opacity=0.6,rectangle,draw}]
\karnaughmap{2}{$z$}{{$x$}{$y$}}
{1011}{
\draw[kmbox] (-0.5,2.5)
   node[below left]{$x$}
   node[above right]{$y$} +(-0.2,0.2)
   node[above left]{$z$};\draw (0,2) -- (-0.7,2.7);
\foreach \x/\1 in %
{0/0,1/1} {
   \node at (\x+0.5,2.2) {\1};
}
\foreach \y/\1 in %
{0/0,1/1} {
   \node at (-0.4,-0.5-\y+2) {\1};
}
}
\end{tikzpicture}
\end{center}
%-------------------------------------------------------------------------------
\subsection{Preenchido com diagramas de Karnaugh com tampas}
\begin{center}
\begin{tikzpicture}[karnaugh,disable bars,x=1\kmunitlength,y=1\kmunitlength,kmbar left sep=1\kmunitlength,grp/.style n args={4}{#1,fill=#1!30,minimum width= #2\kmunitlength,minimum height=#3\kmunitlength,rounded corners=0.2\kmunitlength,fill opacity=0.6,rectangle,draw}]
\karnaughmap{2}{$z$}{{$x$}{$y$}}
{1011}{
\draw[kmbox] (-0.5,2.5)
   node[below left]{$x$}
   node[above right]{$y$} +(-0.2,0.2)
   node[above left]{$z$};\draw (0,2) -- (-0.7,2.7);
\foreach \x/\1 in %
{0/0,1/1} {
   \node at (\x+0.5,2.2) {\1};
}
\foreach \y/\1 in %
{0/0,1/1} {
   \node at (-0.4,-0.5-\y+2) {\1};
}
   \node[grp={LogisimKMapColor1}{0.8}{1.8}](n0) at(0.5,1) {};
   \node[grp={LogisimKMapColor2}{1.8}{0.8}](n1) at(1,0.5) {};
}
\end{tikzpicture}
\end{center}
%===============================================================================
\section{Expressões mínimas}
$z =  \overline{y} +x$~\\
\end{document}
