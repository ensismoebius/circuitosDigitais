\documentclass [15pt,a4paper,twoside]{article}
\usepackage[portuges,shorthands=off]{babel}        % shorhands=off is required for babel french in combination with tikz karnaugh....
\usepackage[utf8x]{inputenc}
\usepackage[T1]{fontenc}
\usepackage{amsmath}
\usepackage{geometry}
\geometry{verbose,a4paper, tmargin=3.5cm,bmargin=3.5cm,lmargin=2.5cm,rmargin=2.5cm,headsep=1cm,footskip=1.5cm}
\usepackage{fancyhdr}
\usepackage{colortbl}
\usepackage[dvipsnames]{xcolor}
\usepackage{tikz -timing}
\usepackage{tikz}
\usetikzlibrary{karnaugh}
\pagestyle{fancy}

\definecolor{LogisimKMapColor0}{RGB}{128,0,0}
\definecolor{LogisimKMapColor1}{RGB}{230,25,75}
\definecolor{LogisimKMapColor2}{RGB}{250,190,190}
\definecolor{LogisimKMapColor3}{RGB}{170,110,40}
\definecolor{LogisimKMapColor4}{RGB}{245,130,48}
\definecolor{LogisimKMapColor5}{RGB}{255,215,180}
\definecolor{LogisimKMapColor6}{RGB}{128,128,0}
\definecolor{LogisimKMapColor7}{RGB}{255,255,25}
\definecolor{LogisimKMapColor8}{RGB}{210,245,60}
\definecolor{LogisimKMapColor9}{RGB}{0,0,128}
\definecolor{LogisimKMapColor10}{RGB}{145,30,180}
\definecolor{LogisimKMapColor11}{RGB}{60,180,175}
\definecolor{LogisimKMapColor12}{RGB}{0,130,203}
\definecolor{LogisimKMapColor13}{RGB}{230,190,255}
\definecolor{LogisimKMapColor14}{RGB}{170,255,195}
\definecolor{LogisimKMapColor15}{RGB}{240,50,230}

\fancyhead{}
\fancyhead[C] {Logisim-evolução gerada pelo documento em Sun Aug 11 19:03:01 BRT 2024}
\fancyfoot[C] {\thepage}
\renewcommand{\headrulewidth}{0.4pt}
\renewcommand{\footrulewidth}{0.4pt}

\makeatother

\begin{document}
\section{Introdução}
Este documento foi gerado por logisim-evolution. Qualquer parte das fontes do TeX pode ser usada em seus próprios documentos sem nenhum problema. Caso você queira usar todas/partes das fontes TeX geradas, por favor (1) não se esqueça de incluir os pacotes necessários, e (2) inclua uma observação de que esta fonte foi gerada pela logisim-evolução.
%===============================================================================
\section{Tabela da verdade}
A tabela pode ser muito grande para ser exibida na página. No tempo de geração não foi feito nenhum cálculo sobre o tamanho da tabela em relação à largura/altura da página.
%-------------------------------------------------------------------------------
\subsection{Tabela da verdade compactada}
\begin{center}
\begin{tabular}{cccc|cc}
$x$&$y$&$z$&$k$&$l$&$m$\\
\hline
$0$&$-$&$0$&$0$&$0$&$0$\\
$0$&$0$&$0$&$1$&$0$&$0$\\
$0$&$0$&$1$&$-$&$1$&$0$\\
$0$&$1$&$0$&$1$&$0$&$1$\\
$-$&$1$&$1$&$0$&$1$&$1$\\
$0$&$1$&$1$&$1$&$1$&$1$\\
$1$&$0$&$0$&$0$&$0$&$0$\\
$1$&$-$&$0$&$1$&$1$&$0$\\
$1$&$0$&$1$&$0$&$1$&$0$\\
$1$&$0$&$1$&$1$&$0$&$0$\\
$1$&$1$&$0$&$0$&$0$&$1$\\
$1$&$1$&$1$&$1$&$0$&$1$\\

\end{tabular}
\end{center}
%-------------------------------------------------------------------------------
\subsection{Tabela da verdade completa}
\begin{center}
\begin{tabular}{cccc|cc}
$x$&$y$&$z$&$k$&$l$&$m$\\
\hline
$0$&$0$&$0$&$0$&$0$&$0$\\
$0$&$0$&$0$&$1$&$0$&$0$\\
$0$&$0$&$1$&$0$&$1$&$0$\\
$0$&$0$&$1$&$1$&$1$&$0$\\
$0$&$1$&$0$&$0$&$0$&$0$\\
$0$&$1$&$0$&$1$&$0$&$1$\\
$0$&$1$&$1$&$0$&$1$&$1$\\
$0$&$1$&$1$&$1$&$1$&$1$\\
$1$&$0$&$0$&$0$&$0$&$0$\\
$1$&$0$&$0$&$1$&$1$&$0$\\
$1$&$0$&$1$&$0$&$1$&$0$\\
$1$&$0$&$1$&$1$&$0$&$0$\\
$1$&$1$&$0$&$0$&$0$&$1$\\
$1$&$1$&$0$&$1$&$1$&$0$\\
$1$&$1$&$1$&$0$&$1$&$1$\\
$1$&$1$&$1$&$1$&$0$&$1$\\

\end{tabular}
\end{center}
%===============================================================================
\section{Diagramas de Karnaugh}
Esta secção mostra várias versões dos diagramas de Karnaugh das funções dadas.
%-------------------------------------------------------------------------------
\subsection{Diagramas de Karnaugh vazios}
\begin{center}
\begin{tikzpicture}[karnaugh,disable bars,x=1\kmunitlength,y=1\kmunitlength,kmbar left sep=1\kmunitlength,grp/.style n args={4}{#1,fill=#1!30,minimum width= #2\kmunitlength,minimum height=#3\kmunitlength,rounded corners=0.2\kmunitlength,fill opacity=0.6,rectangle,draw}]
\karnaughmap{4}{$l$}{{$x$}{$z$}{$y$}{$k$}}{}{
\draw[kmbox] (-0.5,4.5)
   node[below left]{$x$, $y$}
   node[above right]{$z$, $k$} +(-0.2,0.2)
   node[above left]{$l$};\draw (0,4) -- (-0.7,4.7);
\foreach \x/\1 in %
{0/00,1/01,2/11,3/10} {
   \node at (\x+0.5,4.2) {\1};
}
\foreach \y/\1 in %
{0/00,1/01,2/11,3/10} {
   \node at (-0.4,-0.5-\y+4) {\1};
}
}
\end{tikzpicture}
\end{center}
\begin{center}
\begin{tikzpicture}[karnaugh,disable bars,x=1\kmunitlength,y=1\kmunitlength,kmbar left sep=1\kmunitlength,grp/.style n args={4}{#1,fill=#1!30,minimum width= #2\kmunitlength,minimum height=#3\kmunitlength,rounded corners=0.2\kmunitlength,fill opacity=0.6,rectangle,draw}]
\karnaughmap{4}{$m$}{{$x$}{$z$}{$y$}{$k$}}{}{
\draw[kmbox] (-0.5,4.5)
   node[below left]{$x$, $y$}
   node[above right]{$z$, $k$} +(-0.2,0.2)
   node[above left]{$m$};\draw (0,4) -- (-0.7,4.7);
\foreach \x/\1 in %
{0/00,1/01,2/11,3/10} {
   \node at (\x+0.5,4.2) {\1};
}
\foreach \y/\1 in %
{0/00,1/01,2/11,3/10} {
   \node at (-0.4,-0.5-\y+4) {\1};
}
}
\end{tikzpicture}
\end{center}
%-------------------------------------------------------------------------------
\subsection{Preenchido com diagramas de Karnaugh}
\begin{center}
\begin{tikzpicture}[karnaugh,disable bars,x=1\kmunitlength,y=1\kmunitlength,kmbar left sep=1\kmunitlength,grp/.style n args={4}{#1,fill=#1!30,minimum width= #2\kmunitlength,minimum height=#3\kmunitlength,rounded corners=0.2\kmunitlength,fill opacity=0.6,rectangle,draw}]
\karnaughmap{4}{$l$}{{$x$}{$z$}{$y$}{$k$}}
{0000111101011010}{
\draw[kmbox] (-0.5,4.5)
   node[below left]{$x$, $y$}
   node[above right]{$z$, $k$} +(-0.2,0.2)
   node[above left]{$l$};\draw (0,4) -- (-0.7,4.7);
\foreach \x/\1 in %
{0/00,1/01,2/11,3/10} {
   \node at (\x+0.5,4.2) {\1};
}
\foreach \y/\1 in %
{0/00,1/01,2/11,3/10} {
   \node at (-0.4,-0.5-\y+4) {\1};
}
}
\end{tikzpicture}
\end{center}
\begin{center}
\begin{tikzpicture}[karnaugh,disable bars,x=1\kmunitlength,y=1\kmunitlength,kmbar left sep=1\kmunitlength,grp/.style n args={4}{#1,fill=#1!30,minimum width= #2\kmunitlength,minimum height=#3\kmunitlength,rounded corners=0.2\kmunitlength,fill opacity=0.6,rectangle,draw}]
\karnaughmap{4}{$m$}{{$x$}{$z$}{$y$}{$k$}}
{0001001100100011}{
\draw[kmbox] (-0.5,4.5)
   node[below left]{$x$, $y$}
   node[above right]{$z$, $k$} +(-0.2,0.2)
   node[above left]{$m$};\draw (0,4) -- (-0.7,4.7);
\foreach \x/\1 in %
{0/00,1/01,2/11,3/10} {
   \node at (\x+0.5,4.2) {\1};
}
\foreach \y/\1 in %
{0/00,1/01,2/11,3/10} {
   \node at (-0.4,-0.5-\y+4) {\1};
}
}
\end{tikzpicture}
\end{center}
%-------------------------------------------------------------------------------
\subsection{Preenchido com diagramas de Karnaugh com tampas}
\begin{center}
\begin{tikzpicture}[karnaugh,disable bars,x=1\kmunitlength,y=1\kmunitlength,kmbar left sep=1\kmunitlength,grp/.style n args={4}{#1,fill=#1!30,minimum width= #2\kmunitlength,minimum height=#3\kmunitlength,rounded corners=0.2\kmunitlength,fill opacity=0.6,rectangle,draw}]
\karnaughmap{4}{$l$}{{$x$}{$z$}{$y$}{$k$}}
{0000111101011010}{
\draw[kmbox] (-0.5,4.5)
   node[below left]{$x$, $y$}
   node[above right]{$z$, $k$} +(-0.2,0.2)
   node[above left]{$l$};\draw (0,4) -- (-0.7,4.7);
\foreach \x/\1 in %
{0/00,1/01,2/11,3/10} {
   \node at (\x+0.5,4.2) {\1};
}
\foreach \y/\1 in %
{0/00,1/01,2/11,3/10} {
   \node at (-0.4,-0.5-\y+4) {\1};
}
   \node[grp={LogisimKMapColor1}{1.8}{1.8}](n0) at(3,3) {};
   \node[grp={LogisimKMapColor2}{0.8}{1.8}](n1) at(1.5,1) {};
   \node[grp={LogisimKMapColor3}{0.8}{3.8}](n2) at(3.5,2) {};
}
\end{tikzpicture}
\end{center}
\begin{center}
\begin{tikzpicture}[karnaugh,disable bars,x=1\kmunitlength,y=1\kmunitlength,kmbar left sep=1\kmunitlength,grp/.style n args={4}{#1,fill=#1!30,minimum width= #2\kmunitlength,minimum height=#3\kmunitlength,rounded corners=0.2\kmunitlength,fill opacity=0.6,rectangle,draw}]
\karnaughmap{4}{$m$}{{$x$}{$z$}{$y$}{$k$}}
{0001001100100011}{
\draw[kmbox] (-0.5,4.5)
   node[below left]{$x$, $y$}
   node[above right]{$z$, $k$} +(-0.2,0.2)
   node[above left]{$m$};\draw (0,4) -- (-0.7,4.7);
\foreach \x/\1 in %
{0/00,1/01,2/11,3/10} {
   \node at (\x+0.5,4.2) {\1};
}
\foreach \y/\1 in %
{0/00,1/01,2/11,3/10} {
   \node at (-0.4,-0.5-\y+4) {\1};
}
   \node[grp={LogisimKMapColor1}{1.8}{0.8}](n0) at(2,2.5) {};
   \node[grp={LogisimKMapColor2}{1.8}{1.8}](n1) at(3,2) {};
   \node[grp={LogisimKMapColor3}{0.8}{0.8}](n2) at(0.5,1.5) {};
   \node[grp={LogisimKMapColor3}{0.8}{0.8}](n3) at(3.5,1.5) {};
}
\end{tikzpicture}
\end{center}
%===============================================================================


% TODO: \usepackage{graphicx} required
\begin{figure}
	\centering
	% Important: If latex complains about unicode characters,
% please use "\usepackage[utf8x]{inputenc}" in your preamble
% You can change the size of the picture by putting it into the construct:
% 1) \resizebox{10cm}{!}{"below picture"} to scale horizontally to 10 cm
% 2) \resizebox{!}{15cm}{"below picture"} to scale vertically to 15 cm
% 3) \resizebox{10cm}{15cm}{"below picture"} a combination of above two
% It is not recomended to use the scale option of the tikzpicture environment.
\begin{tikzpicture}[x=1pt,y=-1pt,line cap=rect]
\def\logisimfontA#1{\fontfamily{cmr}{#1}} % Replaced by logisim, original font was "SansSerif"
\def\logisimfontB#1{\fontfamily{cmtt}{#1}} % Replaced by logisim, original font was "Monospaced"
\definecolor{custcol_0_0_0}{RGB}{0, 0, 0}
\definecolor{custcol_ff_ff_ff}{RGB}{255, 255, 255}
\draw [line width=3.0pt, custcol_0_0_0 ]  (406.0,375.0) -- (426.0,375.0) ;
\draw [line width=3.0pt, custcol_0_0_0 ]  (106.0,15.0) -- (176.0,15.0) ;
\draw [line width=3.0pt, custcol_0_0_0 ]  (206.0,55.0) -- (256.0,55.0) -- (256.0,225.0) -- (306.0,225.0) ;
\draw [line width=3.0pt, custcol_0_0_0 ]  (126.0,185.0) -- (126.0,265.0) -- (306.0,265.0) ;
\draw [line width=3.0pt, custcol_0_0_0 ]  (176.0,55.0) -- (126.0,55.0) -- (126.0,75.0) -- (126.0,185.0) -- (306.0,185.0) ;
\draw [line width=3.0pt, custcol_0_0_0 ]  (306.0,235.0) -- (136.0,235.0) -- (136.0,105.0) -- (136.0,95.0) -- (176.0,95.0) ;
\draw [line width=3.0pt, custcol_0_0_0 ]  (306.0,435.0) -- (266.0,435.0) -- (266.0,285.0) -- (306.0,285.0) ;
\draw [line width=3.0pt, custcol_0_0_0 ]  (206.0,15.0) -- (236.0,15.0) -- (236.0,165.0) -- (306.0,165.0) ;
\draw [line width=3.0pt, custcol_0_0_0 ]  (136.0,235.0) -- (136.0,335.0) -- (306.0,335.0) ;
\draw [line width=3.0pt, custcol_0_0_0 ]  (116.0,325.0) -- (116.0,365.0) -- (306.0,365.0) ;
\draw [line width=3.0pt, custcol_0_0_0 ]  (406.0,225.0) -- (426.0,225.0) ;
\draw [line width=3.0pt, custcol_0_0_0 ]  (106.0,215.0) -- (306.0,215.0) ;
\draw [line width=3.0pt, custcol_0_0_0 ]  (236.0,165.0) -- (236.0,315.0) -- (306.0,315.0) ;
\draw [line width=3.0pt, custcol_0_0_0 ]  (206.0,95.0) -- (266.0,95.0) -- (266.0,285.0) ;
\draw [line width=3.0pt, custcol_0_0_0 ]  (126.0,265.0) -- (126.0,385.0) -- (306.0,385.0) ;
\draw [line width=3.0pt, custcol_0_0_0 ]  (116.0,365.0) -- (116.0,425.0) -- (306.0,425.0) ;
\fill [line width=3.0pt, custcol_0_0_0]  (136.0,105.0) ellipse (5.0 and 5.0 );
\fill [line width=3.0pt, custcol_0_0_0]  (136.0,235.0) ellipse (5.0 and 5.0 );
\fill [line width=3.0pt, custcol_0_0_0]  (106.0,15.0) ellipse (5.0 and 5.0 );
\fill [line width=3.0pt, custcol_0_0_0]  (116.0,365.0) ellipse (5.0 and 5.0 );
\fill [line width=3.0pt, custcol_0_0_0]  (126.0,185.0) ellipse (5.0 and 5.0 );
\fill [line width=3.0pt, custcol_0_0_0]  (126.0,75.0) ellipse (5.0 and 5.0 );
\fill [line width=3.0pt, custcol_0_0_0]  (106.0,215.0) ellipse (5.0 and 5.0 );
\fill [line width=3.0pt, custcol_0_0_0]  (236.0,165.0) ellipse (5.0 and 5.0 );
\fill [line width=3.0pt, custcol_0_0_0]  (126.0,265.0) ellipse (5.0 and 5.0 );
\fill [line width=3.0pt, custcol_0_0_0]  (116.0,325.0) ellipse (5.0 and 5.0 );
\fill [line width=3.0pt, custcol_0_0_0]  (266.0,285.0) ellipse (5.0 and 5.0 );
\draw [line width=2.0pt, custcol_0_0_0] (321.0,340.0) arc (90.0:-90.0:15.0 and 15.0 );
\draw [line width=2.0pt, custcol_0_0_0 ]  (321.0,310.0) -- (307.0,310.0) -- (307.0,340.0) -- (321.0,340.0) ;
\draw [line width=2.0pt, custcol_0_0_0] (321.0,290.0) arc (90.0:-90.0:15.0 and 15.0 );
\draw [line width=2.0pt, custcol_0_0_0 ]  (321.0,260.0) -- (307.0,260.0) -- (307.0,290.0) -- (321.0,290.0) ;
\draw [line width=2.0pt, custcol_0_0_0] (321.0,190.0) arc (90.0:-90.0:15.0 and 15.0 );
\draw [line width=2.0pt, custcol_0_0_0 ]  (321.0,160.0) -- (307.0,160.0) -- (307.0,190.0) -- (321.0,190.0) ;
\draw [line width=2.0pt, custcol_0_0_0] (321.0,390.0) arc (90.0:-90.0:15.0 and 15.0 );
\draw [line width=2.0pt, custcol_0_0_0 ]  (321.0,360.0) -- (307.0,360.0) -- (307.0,390.0) -- (321.0,390.0) ;
\draw [line width=3.0pt, custcol_0_0_0 ]  (336.0,325.0) -- (356.0,325.0) -- (356.0,365.0) -- (376.0,365.0) -- (376.0,365.0) ;
\draw [line width=3.0pt, custcol_0_0_0 ]  (336.0,375.0) -- (376.0,375.0) -- (376.0,375.0) ;
\draw [line width=3.0pt, custcol_0_0_0 ]  (376.0,385.0) -- (376.0,385.0) -- (356.0,385.0) -- (356.0,425.0) -- (336.0,425.0) ;
\draw [line width=2.0pt, custcol_0_0_0 ]  (406.0,375.0) .. controls  (396.0,360.0)  ..  (376.0,360.0) .. controls  (384.0,375.0)  ..  (376.0,390.0) .. controls  (396.0,390.0)  ..  (406.0,375.0) -- cycle ;
\draw [line width=2.0pt, custcol_0_0_0 ]  (196.0,55.0) -- (177.0,48.0) -- (177.0,62.0) -- cycle;
\draw [line width=2.0pt, custcol_0_0_0]  (201.0,55.0) ellipse (4.5 and 4.5 );
\fill [line width=2.0pt, custcol_0_0_0]  (206.0,55.0) ellipse (2.0 and 2.0 );
\fill [line width=2.0pt, custcol_0_0_0]  (176.0,55.0) ellipse (2.0 and 2.0 );
\draw [line width=3.0pt, custcol_0_0_0 ]  (336.0,175.0) -- (356.0,175.0) -- (356.0,215.0) -- (376.0,215.0) -- (376.0,215.0) ;
\draw [line width=3.0pt, custcol_0_0_0 ]  (336.0,225.0) -- (376.0,225.0) -- (376.0,225.0) ;
\draw [line width=3.0pt, custcol_0_0_0 ]  (336.0,275.0) -- (356.0,275.0) -- (356.0,235.0) -- (376.0,235.0) -- (376.0,235.0) ;
\draw [line width=2.0pt, custcol_0_0_0 ]  (406.0,225.0) .. controls  (396.0,210.0)  ..  (376.0,210.0) .. controls  (384.0,225.0)  ..  (376.0,240.0) .. controls  (396.0,240.0)  ..  (406.0,225.0) -- cycle ;
\draw [line width=2.0pt, custcol_0_0_0] (321.0,440.0) arc (90.0:-90.0:15.0 and 15.0 );
\draw [line width=2.0pt, custcol_0_0_0 ]  (321.0,410.0) -- (307.0,410.0) -- (307.0,440.0) -- (321.0,440.0) ;
\draw [line width=2.0pt, custcol_0_0_0 ]  (196.0,95.0) -- (177.0,88.0) -- (177.0,102.0) -- cycle;
\draw [line width=2.0pt, custcol_0_0_0]  (201.0,95.0) ellipse (4.5 and 4.5 );
\fill [line width=2.0pt, custcol_0_0_0]  (206.0,95.0) ellipse (2.0 and 2.0 );
\fill [line width=2.0pt, custcol_0_0_0]  (176.0,95.0) ellipse (2.0 and 2.0 );
\draw [line width=2.0pt, custcol_0_0_0] (321.0,240.0) arc (90.0:-90.0:15.0 and 15.0 );
\draw [line width=2.0pt, custcol_0_0_0 ]  (321.0,210.0) -- (307.0,210.0) -- (307.0,240.0) -- (321.0,240.0) ;
\draw [line width=2.0pt, custcol_0_0_0 ]  (196.0,15.0) -- (177.0,8.0) -- (177.0,22.0) -- cycle;
\draw [line width=2.0pt, custcol_0_0_0]  (201.0,15.0) ellipse (4.5 and 4.5 );
\fill [line width=2.0pt, custcol_0_0_0]  (206.0,15.0) ellipse (2.0 and 2.0 );
\fill [line width=2.0pt, custcol_0_0_0]  (176.0,15.0) ellipse (2.0 and 2.0 );
\draw [line width=3.0pt, custcol_0_0_0 ]  (430.0,375.0) -- (427.0,375.0) ;
\draw [line width=2.0pt, custcol_0_0_0 ]  (456.0,366.0) -- (466.0,375.0) -- (456.0,384.0) -- (432.0,384.0) -- (432.0,366.0) -- cycle;
\logisimfontB{\fontsize{12pt}{12pt}\selectfont\node[inner sep=0, outer sep=0, custcol_0_0_0, anchor=base west] at  (432.0,381.0)  {x1};}
\logisimfontA{\fontsize{16pt}{16pt}\fontseries{bx}\selectfont\node[inner sep=0, outer sep=0, custcol_0_0_0, anchor=base west] at  (468.0,381.0)  {m};}
\fill [line width=2.0pt, custcol_0_0_0]  (426.0,375.0) ellipse (2.0 and 2.0 );
\draw [line width=3.0pt, custcol_0_0_0 ]  (430.0,225.0) -- (427.0,225.0) ;
\draw [line width=2.0pt, custcol_0_0_0 ]  (456.0,216.0) -- (466.0,225.0) -- (456.0,234.0) -- (432.0,234.0) -- (432.0,216.0) -- cycle;
\logisimfontB{\fontsize{12pt}{12pt}\selectfont\node[inner sep=0, outer sep=0, custcol_0_0_0, anchor=base west] at  (432.0,231.0)  {x1};}
\logisimfontA{\fontsize{16pt}{16pt}\fontseries{bx}\selectfont\node[inner sep=0, outer sep=0, custcol_0_0_0, anchor=base west] at  (468.0,231.0)  {l};}
\fill [line width=2.0pt, custcol_0_0_0]  (426.0,225.0) ellipse (2.0 and 2.0 );
\draw [line width=3.0pt, custcol_0_0_0 ]  (71.0,325.0) -- (76.0,325.0) -- (116.0,325.0) -- (306.0,325.0) ;
\draw [line width=2.0pt, custcol_0_0_0 ]  (61.0,334.0) -- (71.0,325.0) -- (61.0,316.0) -- (37.0,316.0) -- (37.0,334.0) -- cycle;
\logisimfontB{\fontsize{12pt}{12pt}\selectfont\node[inner sep=0, outer sep=0, custcol_0_0_0, anchor=base west] at  (42.0,331.0)  {x1};}
\logisimfontA{\fontsize{16pt}{16pt}\fontseries{bx}\selectfont\node[inner sep=0, outer sep=0, custcol_0_0_0, anchor=base west] at  (25.0,331.0)  {y};}
\fill [line width=2.0pt, custcol_0_0_0]  (76.0,325.0) ellipse (2.0 and 2.0 );
\draw [line width=3.0pt, custcol_0_0_0 ]  (51.0,15.0) -- (56.0,15.0) -- (106.0,15.0) -- (106.0,215.0) -- (106.0,415.0) -- (306.0,415.0) ;
\draw [line width=2.0pt, custcol_0_0_0 ]  (41.0,24.0) -- (51.0,15.0) -- (41.0,6.0) -- (17.0,6.0) -- (17.0,24.0) -- cycle;
\logisimfontB{\fontsize{12pt}{12pt}\selectfont\node[inner sep=0, outer sep=0, custcol_0_0_0, anchor=base west] at  (22.0,21.0)  {x1};}
\logisimfontA{\fontsize{16pt}{16pt}\fontseries{bx}\selectfont\node[inner sep=0, outer sep=0, custcol_0_0_0, anchor=base west] at  (5.0,21.0)  {x};}
\fill [line width=2.0pt, custcol_0_0_0]  (56.0,15.0) ellipse (2.0 and 2.0 );
\draw [line width=3.0pt, custcol_0_0_0 ]  (51.0,75.0) -- (56.0,75.0) -- (126.0,75.0) ;
\draw [line width=2.0pt, custcol_0_0_0 ]  (41.0,84.0) -- (51.0,75.0) -- (41.0,66.0) -- (17.0,66.0) -- (17.0,84.0) -- cycle;
\logisimfontB{\fontsize{12pt}{12pt}\selectfont\node[inner sep=0, outer sep=0, custcol_0_0_0, anchor=base west] at  (22.0,81.0)  {x1};}
\logisimfontA{\fontsize{16pt}{16pt}\fontseries{bx}\selectfont\node[inner sep=0, outer sep=0, custcol_0_0_0, anchor=base west] at  (6.0,81.0)  {z};}
\fill [line width=2.0pt, custcol_0_0_0]  (56.0,75.0) ellipse (2.0 and 2.0 );
\draw [line width=3.0pt, custcol_0_0_0 ]  (51.0,105.0) -- (56.0,105.0) -- (136.0,105.0) ;
\draw [line width=2.0pt, custcol_0_0_0 ]  (41.0,114.0) -- (51.0,105.0) -- (41.0,96.0) -- (17.0,96.0) -- (17.0,114.0) -- cycle;
\logisimfontB{\fontsize{12pt}{12pt}\selectfont\node[inner sep=0, outer sep=0, custcol_0_0_0, anchor=base west] at  (22.0,111.0)  {x1};}
\logisimfontA{\fontsize{16pt}{16pt}\fontseries{bx}\selectfont\node[inner sep=0, outer sep=0, custcol_0_0_0, anchor=base west] at  (5.0,111.0)  {k};}
\fill [line width=2.0pt, custcol_0_0_0]  (56.0,105.0) ellipse (2.0 and 2.0 );
\end{tikzpicture}


	\caption{}
	\label{fig:03exe}
\end{figure}



\section{Expressões mínimas}
$l =  \overline{x}  \cdot z+x \cdot  \overline{z}  \cdot k+z \cdot  \overline{k} $~\\
$m =  \overline{x}  \cdot y \cdot k+y \cdot z+x \cdot y \cdot  \overline{k} $~\\
\end{document}
