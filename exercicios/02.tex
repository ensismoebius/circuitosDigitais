\documentclass [15pt,a4paper,twoside]{article}
\usepackage[portuges,shorthands=off]{babel}        % shorhands=off is required for babel french in combination with tikz karnaugh....
\usepackage[utf8x]{inputenc}
\usepackage[T1]{fontenc}
\usepackage{amsmath}
\usepackage{geometry}
\geometry{verbose,a4paper, tmargin=3.5cm,bmargin=3.5cm,lmargin=2.5cm,rmargin=2.5cm,headsep=1cm,footskip=1.5cm}
\usepackage{fancyhdr}
\usepackage{colortbl}
\usepackage[dvipsnames]{xcolor}
\usepackage{tikz -timing}
\usepackage{tikz}
\usetikzlibrary{karnaugh}
\pagestyle{fancy}

\definecolor{LogisimKMapColor0}{RGB}{128,0,0}
\definecolor{LogisimKMapColor1}{RGB}{230,25,75}
\definecolor{LogisimKMapColor2}{RGB}{250,190,190}
\definecolor{LogisimKMapColor3}{RGB}{170,110,40}
\definecolor{LogisimKMapColor4}{RGB}{245,130,48}
\definecolor{LogisimKMapColor5}{RGB}{255,215,180}
\definecolor{LogisimKMapColor6}{RGB}{128,128,0}
\definecolor{LogisimKMapColor7}{RGB}{255,255,25}
\definecolor{LogisimKMapColor8}{RGB}{210,245,60}
\definecolor{LogisimKMapColor9}{RGB}{0,0,128}
\definecolor{LogisimKMapColor10}{RGB}{145,30,180}
\definecolor{LogisimKMapColor11}{RGB}{60,180,175}
\definecolor{LogisimKMapColor12}{RGB}{0,130,203}
\definecolor{LogisimKMapColor13}{RGB}{230,190,255}
\definecolor{LogisimKMapColor14}{RGB}{170,255,195}
\definecolor{LogisimKMapColor15}{RGB}{240,50,230}

\fancyhead{}
\fancyhead[C] {Logisim-evolução gerada pelo documento em Sun Aug 11 18:54:51 BRT 2024}
\fancyfoot[C] {\thepage}
\renewcommand{\headrulewidth}{0.4pt}
\renewcommand{\footrulewidth}{0.4pt}

\makeatother

\begin{document}
\section{Introdução}
Este documento foi gerado por logisim-evolution. Qualquer parte das fontes do TeX pode ser usada em seus próprios documentos sem nenhum problema. Caso você queira usar todas/partes das fontes TeX geradas, por favor (1) não se esqueça de incluir os pacotes necessários, e (2) inclua uma observação de que esta fonte foi gerada pela logisim-evolução.
%===============================================================================
\section{Tabela da verdade}
A tabela pode ser muito grande para ser exibida na página. No tempo de geração não foi feito nenhum cálculo sobre o tamanho da tabela em relação à largura/altura da página.
%-------------------------------------------------------------------------------
\subsection{Tabela da verdade compactada}
\begin{center}
\begin{tabular}{ccc|c}
$x$&$y$&$z$&$k$\\
\hline
$0$&$-$&$0$&$1$\\
$0$&$0$&$1$&$1$\\
$-$&$1$&$1$&$0$\\
$1$&$-$&$0$&$0$\\
$1$&$0$&$1$&$0$\\

\end{tabular}
\end{center}
%-------------------------------------------------------------------------------
\subsection{Tabela da verdade completa}
\begin{center}
\begin{tabular}{ccc|c}
$x$&$y$&$z$&$k$\\
\hline
$0$&$0$&$0$&$1$\\
$0$&$0$&$1$&$1$\\
$0$&$1$&$0$&$1$\\
$0$&$1$&$1$&$0$\\
$1$&$0$&$0$&$0$\\
$1$&$0$&$1$&$0$\\
$1$&$1$&$0$&$0$\\
$1$&$1$&$1$&$0$\\

\end{tabular}
\end{center}
%===============================================================================
\section{Diagramas de Karnaugh}
Esta secção mostra várias versões dos diagramas de Karnaugh das funções dadas.
%-------------------------------------------------------------------------------
\subsection{Diagramas de Karnaugh vazios}
\begin{center}
\begin{tikzpicture}[karnaugh,disable bars,x=1\kmunitlength,y=1\kmunitlength,kmbar left sep=1\kmunitlength,grp/.style n args={4}{#1,fill=#1!30,minimum width= #2\kmunitlength,minimum height=#3\kmunitlength,rounded corners=0.2\kmunitlength,fill opacity=0.6,rectangle,draw}]
\karnaughmap{3}{$k$}{{$y$}{$x$}{$z$}}{}{
\draw[kmbox] (-0.5,2.5)
   node[below left]{$x$}
   node[above right]{$y$, $z$} +(-0.2,0.2)
   node[above left]{$k$};\draw (0,2) -- (-0.7,2.7);
\foreach \x/\1 in %
{0/00,1/01,2/11,3/10} {
   \node at (\x+0.5,2.2) {\1};
}
\foreach \y/\1 in %
{0/0,1/1} {
   \node at (-0.4,-0.5-\y+2) {\1};
}
}
\end{tikzpicture}
\end{center}
%-------------------------------------------------------------------------------
\subsection{Preenchido com diagramas de Karnaugh}
\begin{center}
\begin{tikzpicture}[karnaugh,disable bars,x=1\kmunitlength,y=1\kmunitlength,kmbar left sep=1\kmunitlength,grp/.style n args={4}{#1,fill=#1!30,minimum width= #2\kmunitlength,minimum height=#3\kmunitlength,rounded corners=0.2\kmunitlength,fill opacity=0.6,rectangle,draw}]
\karnaughmap{3}{$k$}{{$y$}{$x$}{$z$}}
{11001000}{
\draw[kmbox] (-0.5,2.5)
   node[below left]{$x$}
   node[above right]{$y$, $z$} +(-0.2,0.2)
   node[above left]{$k$};\draw (0,2) -- (-0.7,2.7);
\foreach \x/\1 in %
{0/00,1/01,2/11,3/10} {
   \node at (\x+0.5,2.2) {\1};
}
\foreach \y/\1 in %
{0/0,1/1} {
   \node at (-0.4,-0.5-\y+2) {\1};
}
}
\end{tikzpicture}
\end{center}
%-------------------------------------------------------------------------------
\subsection{Preenchido com diagramas de Karnaugh com tampas}
\begin{center}
\begin{tikzpicture}[karnaugh,disable bars,x=1\kmunitlength,y=1\kmunitlength,kmbar left sep=1\kmunitlength,grp/.style n args={4}{#1,fill=#1!30,minimum width= #2\kmunitlength,minimum height=#3\kmunitlength,rounded corners=0.2\kmunitlength,fill opacity=0.6,rectangle,draw}]
\karnaughmap{3}{$k$}{{$y$}{$x$}{$z$}}
{11001000}{
\draw[kmbox] (-0.5,2.5)
   node[below left]{$x$}
   node[above right]{$y$, $z$} +(-0.2,0.2)
   node[above left]{$k$};\draw (0,2) -- (-0.7,2.7);
\foreach \x/\1 in %
{0/00,1/01,2/11,3/10} {
   \node at (\x+0.5,2.2) {\1};
}
\foreach \y/\1 in %
{0/0,1/1} {
   \node at (-0.4,-0.5-\y+2) {\1};
}
   \node[grp={LogisimKMapColor1}{1.8}{0.8}](n0) at(1,1.5) {};
   \node[grp={LogisimKMapColor2}{0.8}{0.8}](n1) at(0.5,1.5) {};
   \node[grp={LogisimKMapColor2}{0.8}{0.8}](n2) at(3.5,1.5) {};
}
\end{tikzpicture}
\end{center}
%===============================================================================
\section{Expressões mínimas}
$k =  \overline{x}  \cdot  \overline{y} + \overline{x}  \cdot  \overline{z} $\\

% TODO: \usepackage{graphicx} required
\begin{figure}
	\centering
	% Important: If latex complains about unicode characters,
% please use "\usepackage[utf8x]{inputenc}" in your preamble
% You can change the size of the picture by putting it into the construct:
% 1) \resizebox{10cm}{!}{"below picture"} to scale horizontally to 10 cm
% 2) \resizebox{!}{15cm}{"below picture"} to scale vertically to 15 cm
% 3) \resizebox{10cm}{15cm}{"below picture"} a combination of above two
% It is not recomended to use the scale option of the tikzpicture environment.
\begin{tikzpicture}[x=1pt,y=-1pt,line cap=rect]
\def\logisimfontA#1{\fontfamily{cmr}{#1}} % Replaced by logisim, original font was "SansSerif"
\def\logisimfontB#1{\fontfamily{cmtt}{#1}} % Replaced by logisim, original font was "Monospaced"
\definecolor{custcol_0_0_0}{RGB}{0, 0, 0}
\definecolor{custcol_ff_ff_ff}{RGB}{255, 255, 255}
\draw [line width=3.0pt, custcol_0_0_0 ]  (96.0,35.0) -- (186.0,35.0) ;
\draw [line width=3.0pt, custcol_0_0_0 ]  (96.0,85.0) -- (186.0,85.0) ;
\draw [line width=3.0pt, custcol_0_0_0 ]  (286.0,45.0) -- (306.0,45.0) ;
\draw [line width=3.0pt, custcol_0_0_0 ]  (116.0,15.0) -- (186.0,15.0) ;
\draw [line width=3.0pt, custcol_0_0_0 ]  (96.0,15.0) -- (116.0,15.0) -- (116.0,65.0) -- (186.0,65.0) ;
\fill [line width=3.0pt, custcol_0_0_0]  (116.0,15.0) ellipse (5.0 and 5.0 );
\draw [line width=3.0pt, custcol_0_0_0 ]  (310.0,45.0) -- (307.0,45.0) ;
\draw [line width=2.0pt, custcol_0_0_0 ]  (336.0,36.0) -- (346.0,45.0) -- (336.0,54.0) -- (312.0,54.0) -- (312.0,36.0) -- cycle;
\logisimfontB{\fontsize{12pt}{12pt}\selectfont\node[inner sep=0, outer sep=0, custcol_0_0_0, anchor=base west] at  (312.0,51.0)  {x1};}
\logisimfontA{\fontsize{16pt}{16pt}\fontseries{bx}\selectfont\node[inner sep=0, outer sep=0, custcol_0_0_0, anchor=base west] at  (348.0,51.0)  {k};}
\fill [line width=2.0pt, custcol_0_0_0]  (306.0,45.0) ellipse (2.0 and 2.0 );
\draw [line width=2.0pt, custcol_0_0_0] (201.0,40.0) arc (90.0:-90.0:15.0 and 15.0 );
\draw [line width=2.0pt, custcol_0_0_0 ]  (201.0,10.0) -- (187.0,10.0) -- (187.0,40.0) -- (201.0,40.0) ;
\draw [line width=3.0pt, custcol_0_0_0 ]  (216.0,25.0) -- (236.0,25.0) -- (236.0,35.0) -- (256.0,35.0) -- (258.0,35.0) ;
\draw [line width=3.0pt, custcol_0_0_0 ]  (216.0,75.0) -- (236.0,75.0) -- (236.0,55.0) -- (256.0,55.0) -- (258.0,55.0) ;
\draw [line width=2.0pt, custcol_0_0_0 ]  (286.0,45.0) .. controls  (276.0,30.0)  ..  (256.0,30.0) .. controls  (264.0,45.0)  ..  (256.0,60.0) .. controls  (276.0,60.0)  ..  (286.0,45.0) -- cycle ;
\draw [line width=2.0pt, custcol_0_0_0] (201.0,90.0) arc (90.0:-90.0:15.0 and 15.0 );
\draw [line width=2.0pt, custcol_0_0_0 ]  (201.0,60.0) -- (187.0,60.0) -- (187.0,90.0) -- (201.0,90.0) ;
\draw [line width=2.0pt, custcol_0_0_0 ]  (86.0,15.0) -- (67.0,8.0) -- (67.0,22.0) -- cycle;
\draw [line width=2.0pt, custcol_0_0_0]  (91.0,15.0) ellipse (4.5 and 4.5 );
\fill [line width=2.0pt, custcol_0_0_0]  (96.0,15.0) ellipse (2.0 and 2.0 );
\fill [line width=2.0pt, custcol_0_0_0]  (66.0,15.0) ellipse (2.0 and 2.0 );
\draw [line width=2.0pt, custcol_0_0_0 ]  (86.0,35.0) -- (67.0,28.0) -- (67.0,42.0) -- cycle;
\draw [line width=2.0pt, custcol_0_0_0]  (91.0,35.0) ellipse (4.5 and 4.5 );
\fill [line width=2.0pt, custcol_0_0_0]  (96.0,35.0) ellipse (2.0 and 2.0 );
\fill [line width=2.0pt, custcol_0_0_0]  (66.0,35.0) ellipse (2.0 and 2.0 );
\draw [line width=3.0pt, custcol_0_0_0 ]  (51.0,35.0) -- (56.0,35.0) -- (66.0,35.0) ;
\draw [line width=2.0pt, custcol_0_0_0 ]  (41.0,44.0) -- (51.0,35.0) -- (41.0,26.0) -- (17.0,26.0) -- (17.0,44.0) -- cycle;
\logisimfontB{\fontsize{12pt}{12pt}\selectfont\node[inner sep=0, outer sep=0, custcol_0_0_0, anchor=base west] at  (22.0,41.0)  {x1};}
\logisimfontA{\fontsize{16pt}{16pt}\fontseries{bx}\selectfont\node[inner sep=0, outer sep=0, custcol_0_0_0, anchor=base west] at  (5.0,41.0)  {y};}
\fill [line width=2.0pt, custcol_0_0_0]  (56.0,35.0) ellipse (2.0 and 2.0 );
\draw [line width=3.0pt, custcol_0_0_0 ]  (51.0,15.0) -- (56.0,15.0) -- (66.0,15.0) ;
\draw [line width=2.0pt, custcol_0_0_0 ]  (41.0,24.0) -- (51.0,15.0) -- (41.0,6.0) -- (17.0,6.0) -- (17.0,24.0) -- cycle;
\logisimfontB{\fontsize{12pt}{12pt}\selectfont\node[inner sep=0, outer sep=0, custcol_0_0_0, anchor=base west] at  (22.0,21.0)  {x1};}
\logisimfontA{\fontsize{16pt}{16pt}\fontseries{bx}\selectfont\node[inner sep=0, outer sep=0, custcol_0_0_0, anchor=base west] at  (5.0,21.0)  {x};}
\fill [line width=2.0pt, custcol_0_0_0]  (56.0,15.0) ellipse (2.0 and 2.0 );
\draw [line width=2.0pt, custcol_0_0_0 ]  (86.0,85.0) -- (67.0,78.0) -- (67.0,92.0) -- cycle;
\draw [line width=2.0pt, custcol_0_0_0]  (91.0,85.0) ellipse (4.5 and 4.5 );
\fill [line width=2.0pt, custcol_0_0_0]  (96.0,85.0) ellipse (2.0 and 2.0 );
\fill [line width=2.0pt, custcol_0_0_0]  (66.0,85.0) ellipse (2.0 and 2.0 );
\draw [line width=3.0pt, custcol_0_0_0 ]  (51.0,85.0) -- (56.0,85.0) -- (66.0,85.0) ;
\draw [line width=2.0pt, custcol_0_0_0 ]  (41.0,94.0) -- (51.0,85.0) -- (41.0,76.0) -- (17.0,76.0) -- (17.0,94.0) -- cycle;
\logisimfontB{\fontsize{12pt}{12pt}\selectfont\node[inner sep=0, outer sep=0, custcol_0_0_0, anchor=base west] at  (22.0,91.0)  {x1};}
\logisimfontA{\fontsize{16pt}{16pt}\fontseries{bx}\selectfont\node[inner sep=0, outer sep=0, custcol_0_0_0, anchor=base west] at  (6.0,91.0)  {z};}
\fill [line width=2.0pt, custcol_0_0_0]  (56.0,85.0) ellipse (2.0 and 2.0 );
\end{tikzpicture}


	\caption{}
	\label{fig:02exe}
\end{figure}

\end{document}
