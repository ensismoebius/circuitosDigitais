\documentclass [15pt,a4paper,twoside]{article}
\usepackage[portuges,shorthands=off]{babel}
\usepackage{graphicx}        % shorhands=off is required for babel french in combination with tikz karnaugh....
\usepackage[utf8x]{inputenc}
\usepackage[T1]{fontenc}
\usepackage{amsmath}
\usepackage{geometry}
\geometry{verbose,a4paper, tmargin=3.5cm,bmargin=3.5cm,lmargin=2.5cm,rmargin=2.5cm,headsep=1cm,footskip=1.5cm}
\usepackage{fancyhdr}
\usepackage{colortbl}
\usepackage[dvipsnames]{xcolor}
\usepackage{tikz -timing}
\usepackage{tikz}
\usetikzlibrary{karnaugh}
\pagestyle{fancy}

\definecolor{LogisimKMapColor0}{RGB}{128,0,0}
\definecolor{LogisimKMapColor1}{RGB}{230,25,75}
\definecolor{LogisimKMapColor2}{RGB}{250,190,190}
\definecolor{LogisimKMapColor3}{RGB}{170,110,40}
\definecolor{LogisimKMapColor4}{RGB}{245,130,48}
\definecolor{LogisimKMapColor5}{RGB}{255,215,180}
\definecolor{LogisimKMapColor6}{RGB}{128,128,0}
\definecolor{LogisimKMapColor7}{RGB}{255,255,25}
\definecolor{LogisimKMapColor8}{RGB}{210,245,60}
\definecolor{LogisimKMapColor9}{RGB}{0,0,128}
\definecolor{LogisimKMapColor10}{RGB}{145,30,180}
\definecolor{LogisimKMapColor11}{RGB}{60,180,175}
\definecolor{LogisimKMapColor12}{RGB}{0,130,203}
\definecolor{LogisimKMapColor13}{RGB}{230,190,255}
\definecolor{LogisimKMapColor14}{RGB}{170,255,195}
\definecolor{LogisimKMapColor15}{RGB}{240,50,230}

\fancyhead{}
\fancyhead[C] {Logisim-evolução gerada pelo documento em Sun Aug 11 18:34:57 BRT 2024}
\fancyfoot[C] {\thepage}
\renewcommand{\headrulewidth}{0.4pt}
\renewcommand{\footrulewidth}{0.4pt}

\makeatother

\begin{document}
\section{Introdução}
Este documento foi gerado por logisim-evolution. Qualquer parte das fontes do TeX pode ser usada em seus próprios documentos sem nenhum problema. Caso você queira usar todas/partes das fontes TeX geradas, por favor (1) não se esqueça de incluir os pacotes necessários, e (2) inclua uma observação de que esta fonte foi gerada pela logisim-evolução.
%===============================================================================
\section{Tabela da verdade}
A tabela pode ser muito grande para ser exibida na página. No tempo de geração não foi feito nenhum cálculo sobre o tamanho da tabela em relação à largura/altura da página.
%-------------------------------------------------------------------------------
\subsection{Tabela da verdade compactada}
\begin{center}
\begin{tabular}{ccc|c}
$a$&$b$&$c$&$d$\\
\hline
$-$&$0$&$-$&$0$\\
$-$&$1$&$0$&$0$\\
$-$&$1$&$1$&$1$\\

\end{tabular}
\end{center}
%-------------------------------------------------------------------------------
\subsection{Tabela da verdade completa}
\begin{center}
\begin{tabular}{ccc|c}
$a$&$b$&$c$&$d$\\
\hline
$0$&$0$&$0$&$0$\\
$0$&$0$&$1$&$0$\\
$0$&$1$&$0$&$0$\\
$0$&$1$&$1$&$1$\\
$1$&$0$&$0$&$0$\\
$1$&$0$&$1$&$0$\\
$1$&$1$&$0$&$0$\\
$1$&$1$&$1$&$1$\\

\end{tabular}
\end{center}
%===============================================================================
\section{Diagramas de Karnaugh}
Esta secção mostra várias versões dos diagramas de Karnaugh das funções dadas.
%-------------------------------------------------------------------------------
\subsection{Diagramas de Karnaugh vazios}
\begin{center}
\begin{tikzpicture}[karnaugh,disable bars,x=1\kmunitlength,y=1\kmunitlength,kmbar left sep=1\kmunitlength,grp/.style n args={4}{#1,fill=#1!30,minimum width= #2\kmunitlength,minimum height=#3\kmunitlength,rounded corners=0.2\kmunitlength,fill opacity=0.6,rectangle,draw}]
\karnaughmap{3}{$d$}{{$b$}{$a$}{$c$}}{}{
\draw[kmbox] (-0.5,2.5)
   node[below left]{$a$}
   node[above right]{$b$, $c$} +(-0.2,0.2)
   node[above left]{$d$};\draw (0,2) -- (-0.7,2.7);
\foreach \x/\1 in %
{0/00,1/01,2/11,3/10} {
   \node at (\x+0.5,2.2) {\1};
}
\foreach \y/\1 in %
{0/0,1/1} {
   \node at (-0.4,-0.5-\y+2) {\1};
}
}
\end{tikzpicture}
\end{center}
%-------------------------------------------------------------------------------
\subsection{Preenchido com diagramas de Karnaugh}
\begin{center}
\begin{tikzpicture}[karnaugh,disable bars,x=1\kmunitlength,y=1\kmunitlength,kmbar left sep=1\kmunitlength,grp/.style n args={4}{#1,fill=#1!30,minimum width= #2\kmunitlength,minimum height=#3\kmunitlength,rounded corners=0.2\kmunitlength,fill opacity=0.6,rectangle,draw}]
\karnaughmap{3}{$d$}{{$b$}{$a$}{$c$}}
{00000101}{
\draw[kmbox] (-0.5,2.5)
   node[below left]{$a$}
   node[above right]{$b$, $c$} +(-0.2,0.2)
   node[above left]{$d$};\draw (0,2) -- (-0.7,2.7);
\foreach \x/\1 in %
{0/00,1/01,2/11,3/10} {
   \node at (\x+0.5,2.2) {\1};
}
\foreach \y/\1 in %
{0/0,1/1} {
   \node at (-0.4,-0.5-\y+2) {\1};
}
}
\end{tikzpicture}
\end{center}
%-------------------------------------------------------------------------------
\subsection{Preenchido com diagramas de Karnaugh com tampas}
\begin{center}
\begin{tikzpicture}[karnaugh,disable bars,x=1\kmunitlength,y=1\kmunitlength,kmbar left sep=1\kmunitlength,grp/.style n args={4}{#1,fill=#1!30,minimum width= #2\kmunitlength,minimum height=#3\kmunitlength,rounded corners=0.2\kmunitlength,fill opacity=0.6,rectangle,draw}]
\karnaughmap{3}{$d$}{{$b$}{$a$}{$c$}}
{00000101}{
\draw[kmbox] (-0.5,2.5)
   node[below left]{$a$}
   node[above right]{$b$, $c$} +(-0.2,0.2)
   node[above left]{$d$};\draw (0,2) -- (-0.7,2.7);
\foreach \x/\1 in %
{0/00,1/01,2/11,3/10} {
   \node at (\x+0.5,2.2) {\1};
}
\foreach \y/\1 in %
{0/0,1/1} {
   \node at (-0.4,-0.5-\y+2) {\1};
}
   \node[grp={LogisimKMapColor1}{0.8}{1.8}](n0) at(2.5,1) {};
}
\end{tikzpicture}
\end{center}
%===============================================================================
\section{Expressões mínimas}
$d = b \cdot c$~\\

\begin{figure}
	\centering
	% Important: If latex complains about unicode characters,
% please use "\usepackage[utf8x]{inputenc}" in your preamble
% You can change the size of the picture by putting it into the construct:
% 1) \resizebox{10cm}{!}{"below picture"} to scale horizontally to 10 cm
% 2) \resizebox{!}{15cm}{"below picture"} to scale vertically to 15 cm
% 3) \resizebox{10cm}{15cm}{"below picture"} a combination of above two
% It is not recomended to use the scale option of the tikzpicture environment.

\resizebox{15cm}{!}{
	\begin{tikzpicture}[x=1pt,y=-1pt,line cap=rect]
	\def\logisimfontA#1{\fontfamily{cmr}{#1}} % Replaced by logisim, original font was "SansSerif"
	\def\logisimfontB#1{\fontfamily{cmtt}{#1}} % Replaced by logisim, original font was "Monospaced"
	\definecolor{custcol_0_0_0}{RGB}{0, 0, 0}
	\definecolor{custcol_ff_ff_ff}{RGB}{255, 255, 255}
	\draw [line width=3.0pt, custcol_0_0_0 ]  (487.0,165.0) -- (527.0,165.0) ;
	\draw [line width=3.0pt, custcol_0_0_0 ]  (77.0,55.0) -- (77.0,145.0) ;
	\draw [line width=3.0pt, custcol_0_0_0 ]  (347.0,145.0) -- (387.0,145.0) ;
	\draw [line width=3.0pt, custcol_0_0_0 ]  (87.0,195.0) -- (507.0,195.0) -- (507.0,185.0) -- (527.0,185.0) ;
	\draw [line width=3.0pt, custcol_0_0_0 ]  (627.0,185.0) -- (647.0,185.0) -- (647.0,205.0) -- (667.0,205.0) ;
	\draw [line width=3.0pt, custcol_0_0_0 ]  (697.0,215.0) -- (717.0,215.0) -- (717.0,15.0) -- (737.0,15.0) ;
	\draw [line width=3.0pt, custcol_0_0_0 ]  (87.0,175.0) -- (367.0,175.0) -- (367.0,165.0) -- (387.0,165.0) ;
	\draw [line width=3.0pt, custcol_0_0_0 ]  (127.0,15.0) -- (67.0,15.0) -- (67.0,125.0) ;
	\draw [line width=3.0pt, custcol_0_0_0 ]  (247.0,145.0) -- (77.0,145.0) -- (77.0,215.0) -- (667.0,215.0) ;
	\fill [line width=3.0pt, custcol_0_0_0]  (67.0,15.0) ellipse (5.0 and 5.0 );
	\fill [line width=3.0pt, custcol_0_0_0]  (87.0,175.0) ellipse (5.0 and 5.0 );
	\fill [line width=3.0pt, custcol_0_0_0]  (87.0,195.0) ellipse (5.0 and 5.0 );
	\fill [line width=3.0pt, custcol_0_0_0]  (77.0,145.0) ellipse (5.0 and 5.0 );
	\fill [line width=3.0pt, custcol_0_0_0]  (77.0,55.0) ellipse (5.0 and 5.0 );
	\fill [line width=3.0pt, custcol_0_0_0]  (67.0,125.0) ellipse (5.0 and 5.0 );
	\draw [line width=3.0pt, custcol_0_0_0 ]  (417.0,155.0) -- (457.0,155.0) -- (457.0,155.0) ;
	\draw [line width=3.0pt, custcol_0_0_0 ]  (157.0,55.0) -- (197.0,55.0) -- (197.0,185.0) -- (437.0,185.0) -- (437.0,175.0) -- (457.0,175.0) -- (457.0,175.0) ;
	\draw [line width=2.0pt, custcol_0_0_0 ]  (487.0,165.0) .. controls  (477.0,150.0)  ..  (457.0,150.0) .. controls  (465.0,165.0)  ..  (457.0,180.0) .. controls  (477.0,180.0)  ..  (487.0,165.0) -- cycle ;
	\draw [line width=3.0pt, custcol_0_0_0 ]  (52.0,75.0) -- (57.0,75.0) -- (87.0,75.0) -- (87.0,175.0) -- (87.0,195.0) -- (87.0,225.0) -- (667.0,225.0) ;
	\draw [line width=2.0pt, custcol_0_0_0 ]  (42.0,84.0) -- (52.0,75.0) -- (42.0,66.0) -- (18.0,66.0) -- (18.0,84.0) -- cycle;
	\logisimfontB{\fontsize{12pt}{12pt}\selectfont\node[inner sep=0, outer sep=0, custcol_0_0_0, anchor=base west] at  (23.0,81.0)  {x1};}
	\logisimfontA{\fontsize{16pt}{16pt}\fontseries{bx}\selectfont\node[inner sep=0, outer sep=0, custcol_0_0_0, anchor=base west] at  (6.0,81.0)  {c};}
	\fill [line width=2.0pt, custcol_0_0_0]  (57.0,75.0) ellipse (2.0 and 2.0 );
	\draw [line width=2.0pt, custcol_0_0_0 ]  (147.0,55.0) -- (128.0,48.0) -- (128.0,62.0) -- cycle;
	\draw [line width=2.0pt, custcol_0_0_0]  (152.0,55.0) ellipse (4.5 and 4.5 );
	\fill [line width=2.0pt, custcol_0_0_0]  (157.0,55.0) ellipse (2.0 and 2.0 );
	\fill [line width=2.0pt, custcol_0_0_0]  (127.0,55.0) ellipse (2.0 and 2.0 );
	\draw [line width=2.0pt, custcol_0_0_0 ]  (147.0,15.0) -- (128.0,8.0) -- (128.0,22.0) -- cycle;
	\draw [line width=2.0pt, custcol_0_0_0]  (152.0,15.0) ellipse (4.5 and 4.5 );
	\fill [line width=2.0pt, custcol_0_0_0]  (157.0,15.0) ellipse (2.0 and 2.0 );
	\fill [line width=2.0pt, custcol_0_0_0]  (127.0,15.0) ellipse (2.0 and 2.0 );
	\draw [line width=2.0pt, custcol_0_0_0] (542.0,190.0) arc (90.0:-90.0:15.0 and 15.0 );
	\draw [line width=2.0pt, custcol_0_0_0 ]  (542.0,160.0) -- (528.0,160.0) -- (528.0,190.0) -- (542.0,190.0) ;
	\draw [line width=2.0pt, custcol_0_0_0] (682.0,230.0) arc (90.0:-90.0:15.0 and 15.0 );
	\draw [line width=2.0pt, custcol_0_0_0 ]  (682.0,200.0) -- (668.0,200.0) -- (668.0,230.0) -- (682.0,230.0) ;
	\draw [line width=2.0pt, custcol_0_0_0] (262.0,150.0) arc (90.0:-90.0:15.0 and 15.0 );
	\draw [line width=2.0pt, custcol_0_0_0 ]  (262.0,120.0) -- (248.0,120.0) -- (248.0,150.0) -- (262.0,150.0) ;
	\draw [line width=3.0pt, custcol_0_0_0 ]  (52.0,15.0) -- (57.0,15.0) -- (67.0,15.0) ;
	\draw [line width=2.0pt, custcol_0_0_0 ]  (42.0,24.0) -- (52.0,15.0) -- (42.0,6.0) -- (18.0,6.0) -- (18.0,24.0) -- cycle;
	\logisimfontB{\fontsize{12pt}{12pt}\selectfont\node[inner sep=0, outer sep=0, custcol_0_0_0, anchor=base west] at  (23.0,21.0)  {x1};}
	\logisimfontA{\fontsize{16pt}{16pt}\fontseries{bx}\selectfont\node[inner sep=0, outer sep=0, custcol_0_0_0, anchor=base west] at  (6.0,21.0)  {a};}
	\fill [line width=2.0pt, custcol_0_0_0]  (57.0,15.0) ellipse (2.0 and 2.0 );
	\draw [line width=3.0pt, custcol_0_0_0 ]  (557.0,175.0) -- (597.0,175.0) -- (597.0,175.0) ;
	\draw [line width=3.0pt, custcol_0_0_0 ]  (247.0,125.0) -- (67.0,125.0) -- (67.0,205.0) -- (577.0,205.0) -- (577.0,195.0) -- (597.0,195.0) -- (597.0,195.0) ;
	\draw [line width=2.0pt, custcol_0_0_0 ]  (627.0,185.0) .. controls  (617.0,170.0)  ..  (597.0,170.0) .. controls  (605.0,185.0)  ..  (597.0,200.0) .. controls  (617.0,200.0)  ..  (627.0,185.0) -- cycle ;
	\draw [line width=3.0pt, custcol_0_0_0 ]  (52.0,45.0) -- (57.0,45.0) -- (77.0,45.0) -- (77.0,55.0) -- (127.0,55.0) ;
	\draw [line width=2.0pt, custcol_0_0_0 ]  (42.0,54.0) -- (52.0,45.0) -- (42.0,36.0) -- (18.0,36.0) -- (18.0,54.0) -- cycle;
	\logisimfontB{\fontsize{12pt}{12pt}\selectfont\node[inner sep=0, outer sep=0, custcol_0_0_0, anchor=base west] at  (23.0,51.0)  {x1};}
	\logisimfontA{\fontsize{16pt}{16pt}\fontseries{bx}\selectfont\node[inner sep=0, outer sep=0, custcol_0_0_0, anchor=base west] at  (5.0,51.0)  {b};}
	\fill [line width=2.0pt, custcol_0_0_0]  (57.0,45.0) ellipse (2.0 and 2.0 );
	\draw [line width=2.0pt, custcol_0_0_0] (402.0,170.0) arc (90.0:-90.0:15.0 and 15.0 );
	\draw [line width=2.0pt, custcol_0_0_0 ]  (402.0,140.0) -- (388.0,140.0) -- (388.0,170.0) -- (402.0,170.0) ;
	\draw [line width=3.0pt, custcol_0_0_0 ]  (741.0,15.0) -- (738.0,15.0) ;
	\draw [line width=2.0pt, custcol_0_0_0 ]  (767.0,6.0) -- (777.0,15.0) -- (767.0,24.0) -- (743.0,24.0) -- (743.0,6.0) -- cycle;
	\logisimfontB{\fontsize{12pt}{12pt}\selectfont\node[inner sep=0, outer sep=0, custcol_0_0_0, anchor=base west] at  (743.0,21.0)  {x1};}
	\logisimfontA{\fontsize{16pt}{16pt}\fontseries{bx}\selectfont\node[inner sep=0, outer sep=0, custcol_0_0_0, anchor=base west] at  (779.0,21.0)  {d};}
	\fill [line width=2.0pt, custcol_0_0_0]  (737.0,15.0) ellipse (2.0 and 2.0 );
	\draw [line width=3.0pt, custcol_0_0_0 ]  (277.0,135.0) -- (317.0,135.0) -- (317.0,135.0) ;
	\draw [line width=3.0pt, custcol_0_0_0 ]  (157.0,15.0) -- (187.0,15.0) -- (187.0,165.0) -- (297.0,165.0) -- (297.0,155.0) -- (317.0,155.0) -- (317.0,155.0) ;
	\draw [line width=2.0pt, custcol_0_0_0 ]  (347.0,145.0) .. controls  (337.0,130.0)  ..  (317.0,130.0) .. controls  (325.0,145.0)  ..  (317.0,160.0) .. controls  (337.0,160.0)  ..  (347.0,145.0) -- cycle ;
	\end{tikzpicture}
}

	\caption{}
	\label{fig:04exe}
\end{figure}

\end{document}
